% To build everything run
%	latexmk -xelatex
% To clean temporary files run
%	latexmk -c
\documentclass[12pt]{scrreprt}

% German locale *************************
\usepackage[T1]{fontenc}
\usepackage[ngerman]{babel}
%****************************************

% Use Verbose citation ******************
%\usepackage[natbibapa]{apacite}
\usepackage{csquotes}
\usepackage{url}
\usepackage[
	backend		= biber,
	style			= verbose,
	autocite	= footnote
	]{biblatex}
	\bibliography{ling-relativ-prinzip}
%****************************************

% Add images ************
\usepackage{graphicx}
%****************************************

% Configure captions ********************
\usepackage[
	format			= hang,
	labelfont		= bf,
	font				= bf,
	figurename	= Abb.,
	tablename		= Tab.
	]{caption}
%****************************************

% Blind text*****************************
\usepackage{lipsum}
%****************************************

% Define itemize bullets ****************
\renewcommand{\labelitemi}{$\bullet$}
\renewcommand{\labelitemii}{$\circ$}
\renewcommand{\labelitemiii}{$\bullet$}
\renewcommand{\labelitemiv}{$\circ$}
%****************************************

% Define enumerate numbers ****************
\renewcommand{\labelenumii}{\arabic{enumi}.\arabic{enumii}}
\renewcommand{\labelenumiii}{\arabic{enumi}.\arabic{enumii}.\arabic{enumiii}.}
\renewcommand{\labelenumiv}{\arabic{enumi}.\arabic{enumii}.\arabic{enumiii}.\arabic{enumiv}.}
%****************************************

% Configure fonts ***********************
% Comment out when fonts are not
% found or use different fonts
%\usepackage{lmodern}
\usepackage{fontspec}
	\setmainfont{TeX Gyre Pagella}
	\setsansfont{TeX Gyre Heros}
%****************************************

% Define layout *************************
\usepackage{setspace}
	\setstretch{1.41}
\usepackage[
	a4paper,
	left		= 2.5cm,
	right		= 3cm,
	top			= 2cm,
	bottom	= 2cm,
	%includeheadfoot,
	%showframe
	]{geometry}
%****************************************

%****************************************
\usepackage{scrlayer-scrpage}
\usepackage[bottom]{footmisc}
	\clearpairofpagestyles

	\setkomafont{pageheadfoot}{\rmfamily}
	%\setkomafont{pagehead}{\bfseries}
	\setkomafont{pagination}{}

	\KOMAoptions{
	   headsepline	= true,
	   footsepline	= false,
	   %plainfootsepline	= true,
	}

	\automark[chapter]{chapter}

	\ihead{\headmark}
	\ohead{\pagemark}
%****************************************

% Make titlepage ************************
\usepackage{titlepage}
	\ititle{Das linguistische Relativitätsprinzip}
	\isubtitle{Sapir-Whorf-Hypothese} % Optional
	\iauthor{Jasper Gude}
	\idate{\today}
	\irefnr{}
	\iaddress{Hockenheim} % Only for ireports!
%****************************************

\begin{document}

\makeititle
%***************
\begin{center}
	\sffamily\bfseries{Eigenständigkeitserklärung}
\end{center}
Ich versichere, dass ich diese Ausarbeitung selbständig verfasst, alle aus anderen Werken
wörtlich oder sinngemäß entnommenen Stellen unter Angabe der Quelle als Entlehnung
kenntlich gemacht und andere als die angegebenen Hilfsmittel nicht benutzt habe.

Oftersheim, \today, Unterschrift
%***************
\tableofcontents
%***************
\listoffigures
%***************
\listoftables
%***************
\chapter{Vorwort}
	\label{chap:vorwort}
\textquote{
\enquote{Wie hängen Sprache, Denken und Wirklichkeit
zusammen?} – Das \enquote{linguistische Relativitätsprinzip} von Benjamin Le
Whorf
\smallskip\newline
Inwiefern ist Sprache ein Medium der Erkenntnis?
Was ist dahingehend die Ansicht der aktuellen Neurowissenschaft?}
\bigskip\newline
Als GFS-Thema hat mich diese Fragestellung sehr angeprochen.
Die Deutschthemen der Klassenstufen davor erschienen mir oft entweder sehr
trocken, oder sehr willkürlich.

Bei diesem Thema gefiel mir direkt der wissenschaftliche Ansatz, der in die
psychologische und neurologische Richtung geht, auch wenn ich zugeben muss, dass
mich diese Bereiche der Wissenschaft zwar interessieren, ich mich damit aber nie
besonders tiefreichend beschäftigt habe.

Ein nettes Nebenprodukt dieser GFS wird also die Erweiterung meines
wissenschaftlichen Horizonts sein.

\chapter{Einleitung}
\label{chap:einleitung}

\chapter{Benjamin Lee Whorf}
\label{chap:bjwhorf}
	\section{Leben und Werk}
	\label{sec:lebenuwerk}
		\subsection{Einflüsse}
		\label{sec:einflüsse}

\chapter{Sapir-Whorf-Hypothese}
\label{chap:sphypothese}
	\section{Interpretationen post mortem}
	\label{sec:interpretpm}
		\subsection{Das Prinzip der linguistischen Relativität}
		\label{sec:lingrelativ}
		\subsection{Das Prinzip des linguistischen Determinismus}
		\label{sec:lingdetermin}
	\section{Aktualität der Hypothese}
	\label{sec:aktualität}
		\subsection{Empirische Forschung}
		\label{sec:empforschung}
		\subsection{Das grammatische Geschlecht}
		\label{sec:gramgeschlecht}

\chapter{Lehnwörter}
\label{chap:lehnwörter}

\begin{table}[!htb]
	\centering
	\caption[Nutzlose Tabelle {\autocite{a}}]{Nutzlose Tabelle\footnotemark}
	\begin{tabular}{ |c|c|c| }
		\hline
		cell1 & cell2 & cell3 \\
		\hline
		cell4 & cell5 & cell6 \\
		cell7 & cell8 & cell9 \\
		\hline
	\end{tabular}
	\label{tab:nutzlos}
\end{table}
\footcitetext{a}

\begin{figure}[!htb]
	\centering
	\includegraphics[width=0.5\textwidth]{example-image-a}
	\caption[Nutzlose Abbildung {\autocite{c}}]{Nutzlose Abbildung\footnotemark}
	\label{fig:nutzlos}
\end{figure}
\footcitetext{c}

\subsubsection{Unterunterabschnittüberschrift}
	\label{sec:unterunterabschnitt}

\begin{itemize}
	\item Lorem ipsum dolor sit amet
	\item Lorem ipsum dolor sit amet
	\begin{itemize}
		\item Lorem ipsum dolor sit amet
		\item Lorem ipsum dolor sit amet
		\item Lorem ipsum dolor sit amet
		\begin{itemize}
			\item Lorem ipsum dolor sit amet
			\item Lorem ipsum dolor sit amet
			\item Lorem ipsum dolor sit amet
		\end{itemize}
	\end{itemize}
	\item Lorem ipsum dolor sit amet
\end{itemize}

\begin{enumerate}
	\item Lorem ipsum dolor sit amet
	\item Lorem ipsum dolor sit amet
	\begin{enumerate}
		\item Lorem ipsum dolor sit amet
		\item Lorem ipsum dolor sit amet
		\item Lorem ipsum dolor sit amet
		\begin{enumerate}
			\item Lorem ipsum dolor sit amet
			\item Lorem ipsum dolor sit amet
			\item Lorem ipsum dolor sit amet
		\end{enumerate}
	\end{enumerate}
	\item Lorem ipsum dolor sit amet
\end{enumerate}

\printbibliography

\end{document}
