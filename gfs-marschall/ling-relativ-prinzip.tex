% To build everything run
%	latexmk -xelatex
% To clean temporary files run
%	latexmk -c
\documentclass[12pt]{scrreprt}

% German locale *************************
\usepackage[T1]{fontenc}
\usepackage[ngerman]{babel}
%****************************************

% Use Verbose citation ******************
%\usepackage[natbibapa]{apacite}
\usepackage{csquotes}
\usepackage{url}
\usepackage[
	backend		= biber,
	style			= verbose,
	autocite	= footnote
	]{biblatex}
	\bibliography{ling-relativ-prinzip}
%****************************************

% Add images ************
\usepackage{graphicx}
%****************************************

% Configure captions ********************
\usepackage[
	format			= hang,
	labelfont		= bf,
	font				= bf,
	figurename	= Abb.,
	tablename		= Tab.
	]{caption}
%****************************************

% Blind text*****************************
\usepackage{lipsum}
%****************************************

% Define itemize bullets ****************
\renewcommand{\labelitemi}{$\bullet$}
\renewcommand{\labelitemii}{$\circ$}
\renewcommand{\labelitemiii}{$\bullet$}
\renewcommand{\labelitemiv}{$\circ$}
%****************************************

% Define enumerate numbers ****************
\renewcommand{\labelenumii}{\arabic{enumi}.\arabic{enumii}}
\renewcommand{\labelenumiii}{\arabic{enumi}.\arabic{enumii}.\arabic{enumiii}.}
\renewcommand{\labelenumiv}{\arabic{enumi}.\arabic{enumii}.\arabic{enumiii}.\arabic{enumiv}.}
%****************************************

% Configure fonts ***********************
% Comment out when fonts are not
% found or use different fonts
%\usepackage{lmodern}
\usepackage{fontspec}
	\setmainfont{TeX Gyre Pagella}
	\setsansfont{TeX Gyre Heros}
%****************************************

% Define layout *************************
\usepackage{setspace}
	\setstretch{1.41}
\usepackage[
	a4paper,
	left		= 2.5cm,
	right		= 3cm,
	top			= 2cm,
	bottom	= 2cm,
	%includeheadfoot,
	%showframe
	]{geometry}
%****************************************

%****************************************
\usepackage{scrlayer-scrpage}
\usepackage[bottom]{footmisc}
	\clearpairofpagestyles

	\setkomafont{pageheadfoot}{\rmfamily}
	%\setkomafont{pagehead}{\bfseries}
	\setkomafont{pagination}{}

	\KOMAoptions{
	   headsepline	= true,
	   footsepline	= false,
	   %plainfootsepline	= true,
	}

	\automark[chapter]{chapter}

	\ihead{\headmark}
	\ohead{\pagemark}
%****************************************

% Make titlepage ************************
\usepackage{titlepage}
	\ititle{Wissenschaftler in der Literatur}
	\isubtitle{Im Vergleich: Wissenschaftler der deutschen Geschichte mit der heutigen Darstellung} % Optional
	\iauthor{Tom Schöchle}
	\idate{\today}
	\irefnr{}
	\iaddress{Hockenheim} % Only for ireports!
%****************************************

\begin{document}

\makeititle
%***************
\begin{center}
	\sffamily\bfseries{Eigenständigkeitserklärung}
\end{center}
Ich versichere, dass ich diese Ausarbeitung selbständig verfasst, alle aus
anderen Werken wörtlich oder sinngemäß entnommenen Stellen unter Angabe der
Quelle als Entlehnung kenntlich gemacht und andere als die angegebenen
Hilfsmittel nicht benutzt habe.

Altlußheim, \today, Unterschrift
%***************
\tableofcontents
%***************
\listoffigures
%***************
\listoftables
%***************
\chapter{Vorwort}
	\label{chap:vorwort}
\blockquote{
\enquote{Wie werden Wissenschaftler in der Literatur dargestellt und verwendet?} – Vom \enquote{Verrückten Wissenschaftler} zum wissbegierigen Doktor Faust.
\smallskip\newline
Dass Wissenschaftler in der Literatur ein sehr gern genutztes Element sind bemerken die meisten Menschen in unserer westlichen Welt bereits in frühem Alter. 
Kinderserien, Kinderbücher, Comics und weitere Medien für Kinder bedienen sich unglaublich gerne an Wissenschaftlern als Charaktere - sowohl als Gegenspieler als auch als Mitstreiter auf der guten Seite.
Gerade hierdurch können sich die meisten Menschen unter einem \enquote{Verrückten Wissenschaftler} direkt etwas vorstellen.

Dass Wissenschaftler auch in klassischen Werken der deutschen Geschichte vorkommen und bewusst verwendet werden ist dementsprechend auch keine Überraschung mehr.
Sowohl Göthe bediente sich bereits in seinem bekanntesten Werk \enquote{Faust} daran, als auch Brecht in seinem Werk \enquote{Leben des Galilei}.
Diese beiden Werke werde ich hauptsächlich nutzen, um die Eigenschaften eines Wissenschaftlers in der Literatur dieser Zeit herauszuarbeiten, um diese am Ende mit denen eines \enquote{Verrückten Wissenschaftlers} per Definition heute zu vergleichen.

Da ich mich mit heutigen Serien, Filmen, oder ähnlichen Medien sehr schlecht auskenne, bleibt der Vergleich aus persönlicher Erfahrung leider aus.
Gerade hier werde ich mich vor allem an der allgemeinen Definition von \enquote{Verrückten Wissenschaftlern} bedienen.
\chapter{Der Verrückte Wissenschaftler}
	\label{chap:der verrückte Wissenschaftler}


\printbibliography

\end{document}
